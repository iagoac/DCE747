\documentclass[compress,mathserif,xcolor=table]{beamer}
\usetheme{sthlm}

%-=-=-=-=-=-=-=-=-=-=-=-=-=-=-=-=-=-=-=-=-=-=-=-=
%        LOADING BEAMER PACKAGES
%-=-=-=-=-=-=-=-=-=-=-=-=-=-=-=-=-=-=-=-=-=-=-=-=

\usepackage{
booktabs,
datetime,
dtk-logos,
graphicx,
multicol,
pgfplots,
ragged2e,
tabularx,
tikz,
wasysym,
multirow,
float,
caption,
subcaption,
amsmath,
mathptmx,
animate
}

\usepackage[scaled=0.9]{helvet}
\usepackage{courier}

\usefonttheme[onlymath]{serif}

\definecolor{mygreen}{RGB}{113, 166, 70}
\definecolor{myblue}{RGB}{68, 140, 185}
\definecolor{myred}{RGB}{217, 98, 55}
\definecolor{mypurple}{RGB}{83, 65, 126}
\definecolor{solviaveis}{RGB}{188, 207, 241}
\definecolor{bronze}{rgb}{0.8, 0.5, 0.2}

\pgfplotsset{compat=1.8}

\usepackage[utf8]{inputenc}
\usepackage[portuguese]{babel}
\usepackage[T1]{fontenc}
\usepackage{newpxtext,newpxmath}
\usepackage{listings}

\lstset{ %
language=[LaTeX]TeX,
basicstyle=\normalsize\ttfamily,
keywordstyle=,
numbers=left,
numberstyle=\tiny\ttfamily,
stepnumber=1,
showspaces=false,
showstringspaces=false,
showtabs=false,
breaklines=true,
frame=tb,
framerule=0.5pt,
tabsize=4,
framexleftmargin=0.5em,
framexrightmargin=0.5em,
xleftmargin=0.5em,
xrightmargin=0.5em
}



%-=-=-=-=-=-=-=-=-=-=-=-=-=-=-=-=-=-=-=-=-=-=-=-=
%        LOADING TIKZ LIBRARIES
%-=-=-=-=-=-=-=-=-=-=-=-=-=-=-=-=-=-=-=-=-=-=-=-=

\usetikzlibrary{
backgrounds,
mindmap
}

%-=-=-=-=-=-=-=-=-=-=-=-=-=-=-=-=-=-=-=-=-=-=-=-=
%        BEAMER OPTIONS
%-=-=-=-=-=-=-=-=-=-=-=-=-=-=-=-=-=-=-=-=-=-=-=-=

\setbeameroption{show notes}

%-=-=-=-=-=-=-=-=-=-=-=-=-=-=-=-=-=-=-=-=-=-=-=-=
%        BEAMER COMMANDS
%-=-=-=-=-=-=-=-=-=-=-=-=-=-=-=-=-=-=-=-=-=-=-=-=


%-=-=-=-=-=-=-=-=-=-=-=-=-=-=-=-=-=-=-=-=-=-=-=-=
%
%	PRESENTATION INFORMATION
%
%-=-=-=-=-=-=-=-=-=-=-=-=-=-=-=-=-=-=-=-=-=-=-=-=

\title{Futuro}
\subtitle{DCE747 - Inglês Técnico}
%\date{\small{\jobname}}
\author{\texttt{Iago Carvalho}}
\institute{\texttt{Departamento de Ciência da Computação}}

\hypersetup{
pdfauthor = {Iago A. Carvalho},      
pdfsubject = {Inglês Técnico},
pdfkeywords = {},  
pdfmoddate= {D:\pdfdate},          
pdfcreator = {WriteLaTeX}
}

\begin{document}

\begin{frame}
\titlepage

\end{frame}

%% --------------------------------------------------------

\begin{frame}{Formas do tempo futuro em Inglês}

Em Português, existem diversas formas de expressar o futuro

\begin{enumerate}
    \item Futuro do presente
    \item Futuro do pretérito
    \item Futuro perfeito
    \item Futuro mais-que-perfeito
    \item Futuro composto do subjuntivo
    \item (...)
\end{enumerate}

\vspace{0.5cm}

Em Inglês, as coisas são mais simples

Existem somente três formas de expressar o futuro

\begin{enumerate}
    \item Utilizando o \textit{will}
    \item Utilizando o verbo \textit{to be} + \textit{going to}
    \item Verbo conjugado no gerúndio (-ing) + Advérbio
\end{enumerate}
\end{frame}

%% --------------------------------------------------------

\begin{frame}{Diferentes formas de futuro}

\textbf{Will}: Expressa ações futuras ainda incertas e pouco planejadas

\begin{itemize}
    \item Ações que podem (ou não) ocorrer no futuro
    \item I will clean my house. (Eu vou limpar minha casa.)
\end{itemize}

\vspace{0.15cm}

\textbf{Going to}: Expressa ações provavelmente ocorrerão no futuro
\begin{itemize}
    \item Ações já planejadas e que tem grandes chances de acontecer
    \item I am going to clean my house. (Eu vou limpar minha casa.)
\end{itemize}

\vspace{0.15cm}

\textbf{Gerúndio + advérbio}: Expressar ações que, com toda certeza, ocorrerão no futuro
\begin{itemize}
    \item Esta forma utiliza um advérbio de tempo para indicar quando a ação ocorrerá
    \item I am cleaning my house after work. (Eu vou limpar minha casa esta após o trabalho.)
\end{itemize}

\end{frame}



%% --------------------------------------------------------

\begin{frame}{Futuro com \textit{will}}

O \textit{simple present} com a adição do \textit{will} expressa o tempo futuro

\vspace{0.5cm}

O futuro com a utilização do \textit{will} é construído utilizando
\begin{enumerate}
    \item \textit{will}
    \item Verbo principal (conjugado no infinitivo)
\end{enumerate}

\end{frame}

%% --------------------------------------------------------

\begin{frame}{Futuro com \textit{will}}

\textbf{Forma afirmativa}: Construção mais simples, sem nenhuma modificação

Sujeito + will + verbo principal + (...)
\begin{itemize}
    \item She will walk on the street. (Ela vai caminhar na rua.)
\end{itemize}

\vspace{0.25cm}

\textbf{Forma negativa}: Adiciona-se o \textit{not} após o \textit{will}

Sujeito + will + not + verbo principal + (...)
\begin{itemize}
    \item She will not walk on the street. (Ela não vai caminhar na rua.)
\end{itemize}

\vspace{0.25cm}

\textbf{Forma interrogativa}: O \textit{will} é utilizado no início da frase

Will + sujeito + verbo principal + (...)
\begin{itemize}
    \item Will she walk on the street? (Ela vai caminhar na rua?)
\end{itemize}

\end{frame}

%% --------------------------------------------------------

\begin{frame}{Futuro com \textit{going to}}

Muito parecido com o \textit{will} 
\begin{itemize}
    \item Utiliza-se o verbo \textit{to be}
    \item Troca-se o \textit{will} pelo \textit{going to}
\end{itemize}

\vspace{0.5cm}

Para a construção do futuro utilizando \textit{going to}, utiliza-se

\begin{enumerate}
    \item Verbo auxiliar (to be)
    \item \textit{going to}
    \item Verbo principal (conjugado no infinitivo)
\end{enumerate}

\end{frame}

%% --------------------------------------------------------

\begin{frame}{Futuro com \textit{going to}}

\textbf{Forma afirmativa}: Construção mais simples, sem nenhuma modificação

Sujeito + to be + going to + verbo principal + (...)
\begin{itemize}
    \item She is going to walk on the street. (Ela vai caminhar na rua.)
\end{itemize}

\vspace{0.25cm}

\textbf{Forma negativa}: Adiciona-se o \textit{not} após o verbo \textit{to be}

Sujeito + to be + not + going to + verbo principal + (...)
\begin{itemize}
    \item She is not going to walk on the street. (Ela não vai caminhar na rua.)
\end{itemize}

\vspace{0.25cm}

\textbf{Forma interrogativa}: O \textit{will} é utilizado no início da frase

To be + sujeito + going to + verbo principal + (...)
\begin{itemize}
    \item Is she going to walk on the street? (Ela vai caminhar na rua?)
\end{itemize}

\end{frame}











%% --------------------------------------------------------

\begin{frame}{Futuro com gerúndio + advérbio}

Esta é a maneira mais diferente de expressar o futuro

\vspace{0.5cm}

O futuro com a utilização do gerúndio e advérbio é construído utilizando
\begin{enumerate}
    \item Verbo to be
    \item Verbo principal (conjugado no gerúndio)
    \begin{itemize}
        \item Sufixo -ing
    \end{itemize}
    \item Advérbio de tempo
\end{enumerate}

\end{frame}

%% --------------------------------------------------------

\begin{frame}{Futuro com gerúndio + advérbio}

\textbf{Forma afirmativa}: Construção mais simples, sem modificações

Sujeito + to be + gerúndio + verbo principal + (...) + advérbio
\begin{itemize}
    \item She is walking on the street tomorrow . (Ela vai caminhar na rua amanhã.)
\end{itemize}

\vspace{0.25cm}

\textbf{Forma negativa}: Adiciona-se o \textit{not} após o verbo \textit{to be}

Sujeito + to be + not + gerúndio + verbo principal + (...) + advérbio
\begin{itemize}
    \item She is not not walking on the street tomorrow. (Ela não vai caminhar na rua amanhã.)
\end{itemize}

\vspace{0.25cm}

\textbf{Forma interrogativa}: O verbo \textit{to be} é utilizado no início da frase

To be + sujeito + gerúndio + verbo principal + (...) + advérbio
\begin{itemize}
    \item Is she walking on the street tomorrow? (Ela vai caminhar na rua amanhã?)
\end{itemize}

\end{frame}

\end{document}