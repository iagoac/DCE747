\documentclass[compress,mathserif]{beamer}
\usetheme{sthlm}

%-=-=-=-=-=-=-=-=-=-=-=-=-=-=-=-=-=-=-=-=-=-=-=-=
%        LOADING BEAMER PACKAGES
%-=-=-=-=-=-=-=-=-=-=-=-=-=-=-=-=-=-=-=-=-=-=-=-=

\usepackage{
booktabs,
datetime,
dtk-logos,
graphicx,
multicol,
pgfplots,
ragged2e,
tabularx,
tikz,
wasysym,
multirow,
float,
caption,
subcaption,
amsmath,
mathptmx,
animate
}

\usepackage[scaled=0.9]{helvet}
\usepackage{courier}

\usefonttheme[onlymath]{serif}

\definecolor{mygreen}{RGB}{113, 166, 70}
\definecolor{myblue}{RGB}{68, 140, 185}
\definecolor{myred}{RGB}{217, 98, 55}
\definecolor{mypurple}{RGB}{83, 65, 126}
\definecolor{solviaveis}{RGB}{188, 207, 241}

\pgfplotsset{compat=1.8}

\usepackage[utf8]{inputenc}
\usepackage[portuguese]{babel}
\usepackage[T1]{fontenc}
\usepackage{newpxtext,newpxmath}
\usepackage{listings}

\lstset{ %
language=[LaTeX]TeX,
basicstyle=\normalsize\ttfamily,
keywordstyle=,
numbers=left,
numberstyle=\tiny\ttfamily,
stepnumber=1,
showspaces=false,
showstringspaces=false,
showtabs=false,
breaklines=true,
frame=tb,
framerule=0.5pt,
tabsize=4,
framexleftmargin=0.5em,
framexrightmargin=0.5em,
xleftmargin=0.5em,
xrightmargin=0.5em
}



%-=-=-=-=-=-=-=-=-=-=-=-=-=-=-=-=-=-=-=-=-=-=-=-=
%        LOADING TIKZ LIBRARIES
%-=-=-=-=-=-=-=-=-=-=-=-=-=-=-=-=-=-=-=-=-=-=-=-=

\usetikzlibrary{
backgrounds,
mindmap
}

%-=-=-=-=-=-=-=-=-=-=-=-=-=-=-=-=-=-=-=-=-=-=-=-=
%        BEAMER OPTIONS
%-=-=-=-=-=-=-=-=-=-=-=-=-=-=-=-=-=-=-=-=-=-=-=-=

\setbeameroption{show notes}

%-=-=-=-=-=-=-=-=-=-=-=-=-=-=-=-=-=-=-=-=-=-=-=-=
%        BEAMER COMMANDS
%-=-=-=-=-=-=-=-=-=-=-=-=-=-=-=-=-=-=-=-=-=-=-=-=


%-=-=-=-=-=-=-=-=-=-=-=-=-=-=-=-=-=-=-=-=-=-=-=-=
%
%	PRESENTATION INFORMATION
%
%-=-=-=-=-=-=-=-=-=-=-=-=-=-=-=-=-=-=-=-=-=-=-=-=

\title{Verbos modais, adjetivos \\ em grau comparativo e \\ termos comparativos}
\subtitle{DCE747 - Inglês Técnico}
%\date{\small{\jobname}}
\author{\texttt{Iago Carvalho}}
\institute{\texttt{Departamento de Ciência da Computação}}

\hypersetup{
pdfauthor = {Iago A. Carvalho},      
pdfsubject = {Pesquisa Operacional},
pdfkeywords = {},  
pdfmoddate= {D:\pdfdate},          
pdfcreator = {WriteLaTeX}
}

\begin{document}

\begin{frame}
\titlepage

\end{frame}

%% --------------------------------------------------------

\begin{frame}{Verbos modais}

Os verbos modais são palavras que acompanham os verbos principais da frase para expressar algum sentido específico.

\vspace{0.5cm}

Eles acrescentam uma segunda ideia, que pode ser 
\begin{itemize}
    \item Possibilidade
    \item Habilidade
    \item Obrigação
    \item Permissão
    \item Proibição
    \item Desejos
    \item $\ldots$
\end{itemize} 

\end{frame}

%% --------------------------------------------------------

\begin{frame}{Verbos modais de pedido}

Os principais são \textit{Can} (poder), \textit{Could} (poderia), \textit{Will} (ir – futuro), e \textit{Would} (iria).

\vspace{0.5cm}

\begin{itemize}
    \item Can you help me please?
    \item Will you come to the party?
    \item Could you tell me what time it is?
    \item Would you pass me the salt, please?
\end{itemize}

\end{frame}

%% --------------------------------------------------------

\begin{frame}{Verbos modais de permissão}

Os principais são \textit{Can} (poder), \textit{Could} (poderia) e \textit{May} (poder – informal).

\vspace{0.5cm}

\begin{itemize}
    \item You can borrow my tent.
    \item Can I have some of your chips?
    \item Could I speak to the manager please?
    \item May I use your car this afternoon?
\end{itemize}

\end{frame}

%% --------------------------------------------------------

\begin{frame}{Verbos modais de ajuda}

Os principais são \textit{Can} (poder), \textit{May} (poder – informal), \textit{Shall} (deveria) e \textit{Will} (ir – futuro).

\vspace{0.5cm}

\begin{itemize}
    \item Can I give you a hand?
    \item May I help you?
    \item Shall I answer the phone?
    \item I will drive you home.
\end{itemize}

\end{frame}

%% --------------------------------------------------------

\begin{frame}{Verbos modais de sugestão e conselho}

Os principais são \textit{Should}, \textit{Ought to} e \textit{Shall} (os 3 significam “deveria”), além de \textit{Have to} (ter que)

\vspace{0.5cm}

\begin{itemize}
    \item You should eat more.
    \item You ought to go to the dentist.
    \item Shall we go?
    \item You really have to try the cake – It’s delicious!
\end{itemize}

\end{frame}

%% --------------------------------------------------------

\begin{frame}{Verbos modais de certeza}

Os principais são \textit{Should} (deveria), \textit{Ought to} (deveria), \textit{Must} (dever), \textit{Can’t} (não poder), e \textit{Will} (ir – futuro)

\vspace{0.5cm}

\begin{itemize}
    \item There is his car... So, he must be here!
    \item She’s late – She must have missed the buss.
    \item You can’t go to the party tonight.
    \item It will be summer soon.
    \item I won’t forget you.
\end{itemize}

\end{frame}

%% --------------------------------------------------------

\begin{frame}{Verbos modais de probabilidade}

Os principais são \textit{Should}, \textit{Shought} e \textit{Ought} (deveria)

\vspace{0.5cm}

\begin{itemize}
    \item They should be in New Jersey by now.
    \item I am just tired – I should feel better tomorrow.
\end{itemize}

\end{frame}

%% --------------------------------------------------------

\begin{frame}{Verbos modais de possibilidade}

Os principais são \textit{May} (poder), \textit{Could} (poderia) e \textit{Might} (poder, dever – em termos de possibilidade)

\vspace{0.5cm}

\begin{itemize}
    \item The TV could be in the garage, but I’m not sure.
    \item I might have done that, I can’t remember.
    \item I may have seen this t-shirt for sale.
\end{itemize}

\end{frame}

%% --------------------------------------------------------

\begin{frame}{Verbos modais de necessidade e obrigação}

Os principais são \textit{Need to} (precisar), \textit{Have to} (ter que), \textit{Have got to} (ter que), \textit{Must} (dever):

\vspace{0.5cm}

\begin{itemize}
    \item I need to go home.
    \item I have to call the school today.
    \item I’ve got to up early tomorrow.
    \item You must to learn English to work here.
    \item You’ll have to help me with homework.
    \item You’ll need to learn English before you go to the USA.
\end{itemize}

\end{frame}

%% --------------------------------------------------------

\begin{frame}{Adjetivos em grau comparativo}

Os adjetivos, além de qualificar substantivos, também fazem comparações

\vspace{0.5cm}

Em Inglês, eles tem 3 diferentes graus de comparação
\begin{itemize}
    \item Grau normal (por exemplo, \textit{beautiful})
    \item Grau comparativo (por exemplo, \textit{more beautiful than})
    \item Grau superlativo (por exemplo, \textit{the most beautiful})
\end{itemize}

\vspace{0.5cm}

Existem algumas regras simples para montarmos adjetivos em grau comparativo

\end{frame}

%% --------------------------------------------------------

\begin{frame}{Regra 1}

De modo geral, todos os adjetivos monossílabos recebem o sufixo -er

\vspace{0.5cm}

\begin{itemize}
    \item \textit{taller than} (mais alto que)
    \item \textit{smaller than} (menor que)
    \item \textit{greater than} (maior que)
    \item older than (mais velho que)
    \item \textit{younger than} (mais novo que, mais jovem que)
    \item \textit{richer than} (mais rico que)
    \item \textit{lower than} (mais baixo que)
    \item \textit{cheaper than} (mais barato que)
    \item \textit{narrower than} (mais estreito que)
\end{itemize}

\end{frame}

%% --------------------------------------------------------

\begin{frame}{Regra 2}

Nos adjetivos terminados em -y, retire o -y e acrescente -ier

\vspace{0.5cm}

\begin{itemize}
    \item \textit{easy} $\rightarrow$ \textit{easier than} (mais fácil que)
    \item \textit{pretty} $\rightarrow$ \textit{prettier than} (mais bonita que)
    \item \textit{sunny} $\rightarrow$ \textit{sunnier than} (mais ensolarado que)
    \item \textit{rainy} $\rightarrow$ \textit{rainier than} (mais chuvoso que)
    \item \textit{happy} $\rightarrow$ \textit{happier than} (mais feliz que)
    \item \textit{busy} $\rightarrow$ \textit{busier than} (mais ocupado que)
    \item \textit{shy} $\rightarrow$ \textit{shier than} (mais tímido que)
\end{itemize}

\end{frame}

%% --------------------------------------------------------

\begin{frame}{Regra 3}

Nos adjetivos terminados em uma sequência de consoante+vogal+consoante, dobre a última consoante e acrescente -er

\vspace{0.5cm}

\begin{itemize}
    \item \textit{hot} $\rightarrow$ \textit{hotter than} (mais quente que)
    \item \textit{big} $\rightarrow$ \textit{bigger than} (maior que)
    \item \textit{wet} $\rightarrow$ \textit{wetter than} (mais molhado que)
    \item \textit{sad} $\rightarrow$ \textit{sadder than} (mais triste que)
\end{itemize}

\end{frame}

%% --------------------------------------------------------

\begin{frame}{Regra 4}

Adjetivos polissílabos e alguns dissílabos receberão a palavra more antes deles

\vspace{0.5cm}

\begin{itemize}
    \item \textit{beautiful} $\rightarrow$ \textit{more beautiful than} (mais bonito que)
    \item \textit{interesting} $\rightarrow$ \textit{more interesting than} (mais interessante que)
    \item \textit{active} $\rightarrow$ \textit{more active than} (mais ativo que)
    \item \textit{useful} $\rightarrow$ \textit{more useful than} (mais útil que)
\end{itemize}

\end{frame}

%% --------------------------------------------------------

\begin{frame}{Regra 5}

Não há regra! Alguns adjetivos possuem uma forma especial
\begin{itemize}
    \item Muito parecido ao conceito de verbos irregulares
\end{itemize}

\vspace{0.5cm}

\begin{itemize}
    \item \textit{good} $\rightarrow$ \textit{better than} (melhor que)
    \item \textit{bad} $\rightarrow$ \textit{worse than} (pior que)
    \item \textit{much} $\rightarrow$ \textit{more than} (mais que)
\end{itemize}

\end{frame}


\end{document}