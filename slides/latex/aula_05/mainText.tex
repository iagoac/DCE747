\documentclass[compress,mathserif]{beamer}
\usetheme{sthlm}

%-=-=-=-=-=-=-=-=-=-=-=-=-=-=-=-=-=-=-=-=-=-=-=-=
%        LOADING BEAMER PACKAGES
%-=-=-=-=-=-=-=-=-=-=-=-=-=-=-=-=-=-=-=-=-=-=-=-=

\usepackage{
booktabs,
datetime,
dtk-logos,
graphicx,
multicol,
pgfplots,
ragged2e,
tabularx,
tikz,
wasysym,
multirow,
float,
caption,
subcaption,
amsmath,
mathptmx,
animate
}

\usepackage[scaled=0.9]{helvet}
\usepackage{courier}

\usefonttheme[onlymath]{serif}

\definecolor{mygreen}{RGB}{113, 166, 70}
\definecolor{myblue}{RGB}{68, 140, 185}
\definecolor{myred}{RGB}{217, 98, 55}
\definecolor{mypurple}{RGB}{83, 65, 126}
\definecolor{solviaveis}{RGB}{188, 207, 241}

\pgfplotsset{compat=1.8}

\usepackage[utf8]{inputenc}
\usepackage[portuguese]{babel}
\usepackage[T1]{fontenc}
\usepackage{newpxtext,newpxmath}
\usepackage{listings}

\lstset{ %
language=[LaTeX]TeX,
basicstyle=\normalsize\ttfamily,
keywordstyle=,
numbers=left,
numberstyle=\tiny\ttfamily,
stepnumber=1,
showspaces=false,
showstringspaces=false,
showtabs=false,
breaklines=true,
frame=tb,
framerule=0.5pt,
tabsize=4,
framexleftmargin=0.5em,
framexrightmargin=0.5em,
xleftmargin=0.5em,
xrightmargin=0.5em
}



%-=-=-=-=-=-=-=-=-=-=-=-=-=-=-=-=-=-=-=-=-=-=-=-=
%        LOADING TIKZ LIBRARIES
%-=-=-=-=-=-=-=-=-=-=-=-=-=-=-=-=-=-=-=-=-=-=-=-=

\usetikzlibrary{
backgrounds,
mindmap
}

%-=-=-=-=-=-=-=-=-=-=-=-=-=-=-=-=-=-=-=-=-=-=-=-=
%        BEAMER OPTIONS
%-=-=-=-=-=-=-=-=-=-=-=-=-=-=-=-=-=-=-=-=-=-=-=-=

\setbeameroption{show notes}

%-=-=-=-=-=-=-=-=-=-=-=-=-=-=-=-=-=-=-=-=-=-=-=-=
%        BEAMER COMMANDS
%-=-=-=-=-=-=-=-=-=-=-=-=-=-=-=-=-=-=-=-=-=-=-=-=


%-=-=-=-=-=-=-=-=-=-=-=-=-=-=-=-=-=-=-=-=-=-=-=-=
%
%	PRESENTATION INFORMATION
%
%-=-=-=-=-=-=-=-=-=-=-=-=-=-=-=-=-=-=-=-=-=-=-=-=

\title{Cognatos e falsos\\ cognatos}
\subtitle{DCE747 - Inglês Técnico}
%\date{\small{\jobname}}
\author{\texttt{Iago Carvalho}}
\institute{\texttt{Departamento de Ciência da Computação}}

\hypersetup{
pdfauthor = {Iago A. Carvalho},      
pdfsubject = {Pesquisa Operacional},
pdfkeywords = {},  
pdfmoddate= {D:\pdfdate},          
pdfcreator = {WriteLaTeX}
}

\begin{document}

\begin{frame}
\titlepage

\end{frame}

%% --------------------------------------------------------

\begin{frame}{Cognatos}

São palavras de outros idiomas que são parecidas com outras em nosso idioma
\begin{itemize}
    \item São também chamadas de palavras transparentes
\end{itemize}

\vspace{0.5cm}

Cognatos são divididos em duas categorias

\begin{itemize}
    \item Idênticos
    \item Parecidos
\end{itemize}

\vspace{0.5cm}

A identificação de cognatos nos ajuda a identificar o sentido geral do texto

\end{frame}

%% --------------------------------------------------------

\begin{frame}{Idênticos e parecidos}

Idênticos:
\begin{itemize}
    \item radio, piano, hospital, hotel, sofa, nuclear, social, total, particular, chance, camera, inventor
\end{itemize}

\vspace{0.5cm}

Parecidos
\begin{itemize}
    \item gasoline, banks, inflation, intelligent, population, revolution, commercial, attention, different, products, secretary, billion, dramatic, deposits, distribution, automatic, television, public, events, models, electricity, responsible, explain, activity, impossible, lamp, company
\end{itemize}

\end{frame}

%% --------------------------------------------------------

\begin{frame}{Falsos cognatos}

São palavras com grafia parecidas com as de nosso idioma
\begin{itemize}
    \item Entretanto, tem sentido completamente diferente!
\end{itemize}

\vspace{0.5cm}

Falsos cognatos podem te atrapalhar na leitura de um texto
\begin{itemize}
    \item A grafia parecida pode te induzir ao erro
\end{itemize}

\vspace{0.5cm}

Não existe uma regra simples para determinar se uma palavra é um cognato verdadeiro ou falso
\begin{itemize}
    \item A única coisa que pode te ajudar é um dicionário
\end{itemize}
\end{frame}

%% --------------------------------------------------------

\begin{frame}{Falsos cognatos mais comuns}

\begin{itemize}
    \item He arrived after lunch. (Ele chegou depois do almoço.)
    \item We go to college together. (Nós vamos à faculdade juntos.)
    \item Did you meet my parents? (Você encontrou com meus pais?)
    \item This pasta is delicious! (Este macarrão está delicioso!)
    \item Please, push the door. (Por favor, empurre a porta.)
\end{itemize}

\vspace{0.5cm}

\begin{itemize}
    \item Meu vizinho trabalha em uma fábrica. (My neighbor works in a factory.)
    \item Atenda o telefone! (Answer the phone!)
    \item Assistimos o filme sem legendas. (We watched the movie without subtitles.)
    \item Minha vó adora assistir novelas. (My grandma loves to watch soap operas.)
\end{itemize}

\vspace{0.5cm}

\centering \href{https://www.sk.com.br/sk-falsos-cognatos-ou-falsos-amigos.html}{\beamergotobutton{Link}} \hspace{1cm}
\href{https://www.todamateria.com.br/falsos-cognatos-no-ingles-false-friends/}{\beamergotobutton{Link}}

\end{frame}

%% --------------------------------------------------------

\begin{frame}{Exercício 1}

Roraima is an interesting mountain located in the Guiana Highlands. The peak actually shares the border with Venezuela, Brazil, and Guyana, but the mountain is almost always approached from the Venezuela side. The Brazil and Guyana sides are much more difficult. The mountain’s highest point is Maverick Rock which is at and on the Venezuela side.

Marque a alternativa que contém um \textbf{falso cognato}

\begin{enumerate}[a)]
    \item mountain
    \item interesting
    \item point
    \item actually
\end{enumerate}

\end{frame}

%% --------------------------------------------------------

\begin{frame}{Exercício 1}

Roraima is an interesting mountain located in the Guiana Highlands. The peak actually shares the border with Venezuela, Brazil, and Guyana, but the mountain is almost always approached from the Venezuela side. The Brazil and Guyana sides are much more difficult. The mountain’s highest point is Maverick Rock which is at and on the Venezuela side.

Marque a alternativa que contém um \textbf{falso cognato}

\begin{enumerate}[a)]
    \item mountain
    \item interesting
    \item point
    \item \textit{\textbf{actually}}
\end{enumerate}

\end{frame}

%% --------------------------------------------------------

\begin{frame}{Exercício 2}

Se alguém disser que deseja \textit{a dessert}, o que essa pessoa deseja?

\begin{enumerate}[a)]
    \item Um deserto
    \item Uma sobremesa
    \item Uma viagem
    \item Uma refeição
\end{enumerate}

\end{frame}

%% --------------------------------------------------------

\begin{frame}{Exercício 2}

Se alguém disser que deseja \textit{a dessert}, o que essa pessoa deseja?

\begin{enumerate}[a)]
    \item Um deserto
    \item \textit{\textbf{Uma sobremesa}}
    \item Uma viagem
    \item Uma refeição
\end{enumerate}

\end{frame}

%% --------------------------------------------------------

\begin{frame}{Exercício 3}

Se eu te digo que quero \textit{an avocado}, qual é a sua reação?

\begin{enumerate}[a)]
    \item Sai correndo
    \item Diz que conhece um bom advogado
    \item Oferece um abacate
    \item Diz que não é capaz de invocar nada
    \item Oferece um sanduíche
\end{enumerate}

\end{frame}

%% --------------------------------------------------------

\begin{frame}{Exercício 3}

Se eu te digo que quero \textit{an avocado}, qual é a sua reação?

\begin{enumerate}[a)]
    \item Sai correndo
    \item Diz que conhece um bom advogado
    \item \textit{\textbf{Oferece um abacate}}
    \item Diz que não é capaz de invocar nada
    \item Oferece um sanduíche
\end{enumerate}

\end{frame}

\end{document}