\documentclass[compress,mathserif,xcolor=table]{beamer}
\usetheme{sthlm}

%-=-=-=-=-=-=-=-=-=-=-=-=-=-=-=-=-=-=-=-=-=-=-=-=
%        LOADING BEAMER PACKAGES
%-=-=-=-=-=-=-=-=-=-=-=-=-=-=-=-=-=-=-=-=-=-=-=-=

\usepackage{
booktabs,
datetime,
dtk-logos,
graphicx,
multicol,
pgfplots,
ragged2e,
tabularx,
tikz,
wasysym,
multirow,
float,
caption,
subcaption,
amsmath,
mathptmx,
animate
}

\usepackage[scaled=0.9]{helvet}
\usepackage{courier}

\usefonttheme[onlymath]{serif}

\definecolor{mygreen}{RGB}{113, 166, 70}
\definecolor{myblue}{RGB}{68, 140, 185}
\definecolor{myred}{RGB}{217, 98, 55}
\definecolor{mypurple}{RGB}{83, 65, 126}
\definecolor{solviaveis}{RGB}{188, 207, 241}

\pgfplotsset{compat=1.8}

\usepackage[utf8]{inputenc}
\usepackage[portuguese]{babel}
\usepackage[T1]{fontenc}
\usepackage{newpxtext,newpxmath}
\usepackage{listings}

\lstset{ %
language=[LaTeX]TeX,
basicstyle=\normalsize\ttfamily,
keywordstyle=,
numbers=left,
numberstyle=\tiny\ttfamily,
stepnumber=1,
showspaces=false,
showstringspaces=false,
showtabs=false,
breaklines=true,
frame=tb,
framerule=0.5pt,
tabsize=4,
framexleftmargin=0.5em,
framexrightmargin=0.5em,
xleftmargin=0.5em,
xrightmargin=0.5em
}



%-=-=-=-=-=-=-=-=-=-=-=-=-=-=-=-=-=-=-=-=-=-=-=-=
%        LOADING TIKZ LIBRARIES
%-=-=-=-=-=-=-=-=-=-=-=-=-=-=-=-=-=-=-=-=-=-=-=-=

\usetikzlibrary{
backgrounds,
mindmap
}

%-=-=-=-=-=-=-=-=-=-=-=-=-=-=-=-=-=-=-=-=-=-=-=-=
%        BEAMER OPTIONS
%-=-=-=-=-=-=-=-=-=-=-=-=-=-=-=-=-=-=-=-=-=-=-=-=

\setbeameroption{show notes}

%-=-=-=-=-=-=-=-=-=-=-=-=-=-=-=-=-=-=-=-=-=-=-=-=
%        BEAMER COMMANDS
%-=-=-=-=-=-=-=-=-=-=-=-=-=-=-=-=-=-=-=-=-=-=-=-=


%-=-=-=-=-=-=-=-=-=-=-=-=-=-=-=-=-=-=-=-=-=-=-=-=
%
%	PRESENTATION INFORMATION
%
%-=-=-=-=-=-=-=-=-=-=-=-=-=-=-=-=-=-=-=-=-=-=-=-=

\title{Pronomes}
\subtitle{DCE747 - Inglês Técnico}
%\date{\small{\jobname}}
\author{\texttt{Iago Carvalho}}
\institute{\texttt{Departamento de Ciência da Computação}}

\hypersetup{
pdfauthor = {Iago A. Carvalho},      
pdfsubject = {Pesquisa Operacional},
pdfkeywords = {},  
pdfmoddate= {D:\pdfdate},          
pdfcreator = {WriteLaTeX}
}

\begin{document}

\begin{frame}
\titlepage

\end{frame}

%% --------------------------------------------------------

\begin{frame}{O que é um pronome?}

Pronome é uma classe de palavras variável cuja finalidade é substituir ou determinar (acompanhar) um substantivo

\vspace{0.5cm}

Em Inglês, existem pronomes com diversas funcionalidades
\begin{itemize} 
    \item Pessoais
    \item Objeto
    \item Possessivos
    \item Reflexivos
    \item Demonstrativos
    \item Indefinidos
    \item Relativos
    \item Interrogativos
\end{itemize} 

\end{frame}

%% --------------------------------------------------------

\begin{frame}{Pronomes pessoais}

Equivale ao sujeito em uma sentença, substituindo um substantivo

Pronomes pessoais tem que ser usados em conjunto com um verbo
\begin{itemize}
    \item Como equivale ao sujeito, ele tem que realizar alguma ação
    \item Da mesma forma que um substantivo
\end{itemize}

\vspace{0.5cm}

\begin{itemize}
    \item \textbf{I} live in Brazil. $\rightarrow$ \textbf{Eu} moro no Brasil;
    \item \textbf{You} drink water. $\rightarrow$ \textbf{Você} bebe agua;
    \item \textbf{He} speaks portuguese $\rightarrow$ \textbf{Ele} fala português;
    \item \textbf{She} drives a red car. $\rightarrow$ \textbf{Ela} dirige um carro vermelho;
    \item \textbf{It} bites. $\rightarrow$ \textbf{Ele} morde;
    \item \textbf{We} eat meat. $\rightarrow$ \textbf{Nós} comemos carne;
    \item \textbf{You} read a magazine. $\rightarrow$ \textbf{Vocês} leem uma revista;
    \item \textbf{They} have a house $\rightarrow$ \textbf{Eles/elas} têm uma casa.
\end{itemize}

\end{frame}

%% --------------------------------------------------------

\begin{frame}{Pronomes objeto}

Equivale ao objeto da oração
\begin{itemize}
    \item Quem recebe a ação
    \item Deste modo, sua utilização nunca é conjugada com um verbo
\end{itemize}

\vspace{0.5cm}

\begin{itemize}
    \item Give it to \textbf{me}. / Dê isso para \textbf{mim}.
    \item I will give an apple to \textbf{you}. / Eu vou dar uma maçã para \textbf{você}.
    \item Show \textbf{him} the living room / Mostre a sala de estar a \textbf{ele}.
    \item I gave \textbf{her} a book / Eu dei um livro a \textbf{ela}
    \item She blamed \textbf{us} for the delay. / Ela \textbf{nos} culpou pelo atraso.
    \item I will book \textbf{you} a romm. / Eu vou reservar um quarto para \textbf{vocês}.
    \item Tell \textbf{them} I miss her. / Diga a \textbf{eles} que eu sinto saudade dela.
\end{itemize}

\end{frame}

%% --------------------------------------------------------

\begin{frame}{Pronomes objeto}

\begin{table}[]
\centering
\resizebox{\textwidth}{!}{%
\begin{tabular}{@{}ccc@{}}
\toprule
\rowcolor[HTML]{F9F9F9} 
{\color[HTML]{212529} \textbf{Pronome}} & {\color[HTML]{212529} \textbf{Pronome Objeto}} & {\color[HTML]{212529} \textbf{Tradução}}                                    \\ \midrule
\rowcolor[HTML]{F9F9F9} 
{\color[HTML]{212529} I}   & {\color[HTML]{212529} Me}  & {\color[HTML]{212529} Eu / me / mim / comigo você} \\
\rowcolor[HTML]{F5F5F5} 
{\color[HTML]{212529} You} & {\color[HTML]{212529} You} & {\color[HTML]{212529} Você / te / ti / contigo}    \\
\rowcolor[HTML]{F9F9F9} 
{\color[HTML]{212529} He}  & {\color[HTML]{212529} Him} & {\color[HTML]{212529} Ele / o / lhe / lo / no}     \\
\rowcolor[HTML]{F5F5F5} 
{\color[HTML]{212529} She} & {\color[HTML]{212529} Her} & {\color[HTML]{212529} Ela / a / lhe / la / na}     \\
\rowcolor[HTML]{F9F9F9} 
{\color[HTML]{212529} It}               & {\color[HTML]{212529} It}                      & {\color[HTML]{212529} Ele / ela / o / a / lhe / lo / la / no / na}          \\
\rowcolor[HTML]{F5F5F5} 
{\color[HTML]{212529} We}  & {\color[HTML]{212529} Us}  & {\color[HTML]{212529} Nós / nos / conosco}         \\
\rowcolor[HTML]{F9F9F9} 
{\color[HTML]{212529} You} & {\color[HTML]{212529} You} & {\color[HTML]{212529} Vocês / los / nos / os / as} \\
\rowcolor[HTML]{F5F5F5} 
{\color[HTML]{212529} They}             & {\color[HTML]{212529} Them}                    & {\color[HTML]{212529} Eles / elas / os / as / los / las / nos / nas / lhes} \\ \bottomrule
\end{tabular}%
}
\end{table}

\end{frame}

%% --------------------------------------------------------

\begin{frame}{Pronomes possessivos}

Indicam posse de algo
\begin{itemize}
    \item Sempre utilizados antes de um substantivo
\end{itemize}

\vspace{0.5cm}

\begin{itemize}
    \item \textbf{My} house is white. $\rightarrow$ \textbf{Minha} casa é branca.
    \item \textbf{Your} car is beautiful. $\rightarrow$ \textbf{Seu} carro é bonito.
    \item \textbf{His} phone is very expensive. $\rightarrow$ O celular \textbf{dele} é muito caro.
    \item \textbf{Her} boyfriends is a doctor. $\rightarrow$ O namorado \textbf{dela} é um médico.
    \item \textbf{Our} bathroom is clean. $\rightarrow$ \textbf{Nosso} banheiro é limpo;
    \item \textbf{Your} green eyes are fabulous. $\rightarrow$ \textbf{Seus} olhos verdes são fabuloso;
    \item \textbf{Their} children are healthy. $\rightarrow$ Os filhos \textbf{deles} são saudáveis.
\end{itemize}
\end{frame}

%% --------------------------------------------------------

\begin{frame}{Pronomes possessivos}

\begin{table}[]
\centering
\label{tab:my-table2}
\resizebox{0.7\textwidth}{!}{%
\begin{tabular}{@{}cc@{}}
\toprule
\rowcolor[HTML]{F9F9F9} 
{\color[HTML]{212529} \textbf{Pronome}} & {\color[HTML]{212529} \textbf{Tradução}}              \\ \midrule
\rowcolor[HTML]{F5F5F5} 
{\color[HTML]{212529} My}               & {\color[HTML]{212529} Meu / minha / meus / minhas}    \\
\rowcolor[HTML]{F9F9F9} 
{\color[HTML]{212529} Your}             & {\color[HTML]{212529} Seu / sua}                      \\
\rowcolor[HTML]{F5F5F5} 
{\color[HTML]{212529} His}              & {\color[HTML]{212529} Dele}                           \\
\rowcolor[HTML]{F9F9F9} 
{\color[HTML]{212529} Her}              & {\color[HTML]{212529} dela}                           \\
\rowcolor[HTML]{F5F5F5} 
{\color[HTML]{212529} Its}              & {\color[HTML]{212529} Dele / dela (coisas e animais)} \\
\rowcolor[HTML]{F9F9F9} 
{\color[HTML]{212529} Our} & {\color[HTML]{212529} Nosso / nossa / nossos / nossas} \\
\rowcolor[HTML]{F5F5F5} 
{\color[HTML]{212529} Your}             & {\color[HTML]{212529} Seus / suas}                    \\
\rowcolor[HTML]{F9F9F9} 
{\color[HTML]{212529} Their}            & {\color[HTML]{212529} Deles / delas}                  \\ \bottomrule
\end{tabular}%
}
\end{table}

\end{frame}

%% --------------------------------------------------------

\begin{frame}{Pronomes reflexivos}

Pronomes utilizados quando o sujeito da oração pratica e é alvo de uma ação

\vspace{0.5cm}

\begin{itemize}
    \item  I looked at \textbf{myself} in the photo. $\rightarrow$ Eu \textbf{me} olhei na foto.
    \item How do you describe \textbf{yourself}? $\rightarrow$ Como você descreve a \textbf{si mesmo}?
    \item He does not know \textbf{himself} $\rightarrow$ Ele não \textbf{se} conhece.
    \item She takes care of \textbf{herself}. $\rightarrow$ Ela cuida de \textbf{si mesma}.
    \item The cat is licking \textbf{itself}. $\rightarrow$ O gato está \textbf{se} lambendo.
    \item We love \textbf{ourselves}. $\rightarrow$ Nós \textbf{nos} amamos.
    \item You should love \textbf{yourselves}. $\rightarrow$ Vocês deveriam amar a \textbf{si próprios}.
    \item They are proud of \textbf{themselves}. $\rightarrow$ Eles estão orgulhosos \textbf{deles mesmos}.
\end{itemize}

\end{frame}

%% --------------------------------------------------------

\begin{frame}{Pronomes reflexivos}

\begin{table}[]
\centering
\label{tab:my-table3}
\resizebox{\textwidth}{!}{%
\begin{tabular}{@{}lll@{}}
\toprule
\rowcolor[HTML]{F9F9F9} 
{\color[HTML]{212529} \textbf{Pronome pessoal}} &
  {\color[HTML]{212529} \textbf{Pronome reflexivo}} &
  \multicolumn{1}{c}{\cellcolor[HTML]{F9F9F9}{\color[HTML]{212529} \textbf{Tradução}}} \\ \midrule
\rowcolor[HTML]{F5F5F5} 
{\color[HTML]{212529} I}    & {\color[HTML]{212529} Myself}     & \cellcolor[HTML]{F9F9F9}{\color[HTML]{212529} A mim mesmo, -me}                        \\
\rowcolor[HTML]{F9F9F9} 
{\color[HTML]{212529} You}  & {\color[HTML]{212529} Yourself}   & \cellcolor[HTML]{F5F5F5}{\color[HTML]{212529} A ti, a você mesmo(a), -te,-se}          \\
\rowcolor[HTML]{F5F5F5} 
{\color[HTML]{212529} He}   & {\color[HTML]{212529} Himself}    & \cellcolor[HTML]{F9F9F9}{\color[HTML]{212529} A si, a ele mesmo, -se}                  \\
\rowcolor[HTML]{F9F9F9} 
{\color[HTML]{212529} Her}  & {\color[HTML]{212529} Herself}    & \cellcolor[HTML]{F5F5F5}{\color[HTML]{212529} A si, a ela mesma, -se}                  \\
\rowcolor[HTML]{F5F5F5} 
{\color[HTML]{212529} It}   & {\color[HTML]{212529} Itself}     & \cellcolor[HTML]{F9F9F9}{\color[HTML]{212529} A si mesmo(a), -se}                      \\
\rowcolor[HTML]{F9F9F9} 
{\color[HTML]{212529} We}   & {\color[HTML]{212529} Ourselves}  & \cellcolor[HTML]{F5F5F5}{\color[HTML]{212529} A nós mesmos(as), -nos}                  \\
\rowcolor[HTML]{F5F5F5} 
{\color[HTML]{212529} You}  & {\color[HTML]{212529} Yourselves} & \cellcolor[HTML]{F9F9F9}{\color[HTML]{212529} A vós, a vocês mesmos(as), -vos,-se}     \\
\rowcolor[HTML]{F9F9F9} 
{\color[HTML]{212529} They} & {\color[HTML]{212529} Themselves} & \cellcolor[HTML]{F5F5F5}{\color[HTML]{212529} A si, a eles mesmos, a elas mesmas, -se} \\ \bottomrule
\end{tabular}%
}
\end{table}

\end{frame}

%% --------------------------------------------------------

\begin{frame}{Pronomes demonstrativos}

Têm a função principal de apontar uma pessoa, objeto, lugar, indicando ou mostrando a sua localização. 

\vspace{0.25cm}   

Pode acompanhar um substantivo ou referir-se a ele, por isso, os pronomes demonstrativos fazem flexão de número, ou seja, singular ou plural.

\vspace{0.25cm}   

\begin{itemize}
    \item \textbf{This} is my house. These are our dogs. $\rightarrow$ \textbf{Esta} é a minha casa. \textbf{Estes} são os nossos cachorros.
    \item \textbf{That} store is very far from here. $\rightarrow$ \textbf{Aquela} loja é muito longe daqui.
    \item \textbf{Those} shoes are very beautiful. $\rightarrow$ \textbf{Aqueles} sapatos são muito bonitos. 
\end{itemize}

\end{frame}

%% --------------------------------------------------------

\begin{frame}{Pronomes demonstrativos}

\begin{table}[]
\centering
\label{tab:my-table4}
\resizebox{\textwidth}{!}{%
\begin{tabular}{@{}ll@{}}
\toprule
\rowcolor[HTML]{F9F9F9} 
{\color[HTML]{212529} \textbf{Pronome demonstrativo}} &
  {\color[HTML]{212529} \textbf{Tradução}} \\ \midrule
\rowcolor[HTML]{F5F5F5} 
{\color[HTML]{212529} This (singular)}    & {\color[HTML]{212529} Este, esta, isto}                           \\
\rowcolor[HTML]{F9F9F9} 
{\color[HTML]{212529} That (singular)}  & {\color[HTML]{212529} Esse, essa, isso, aquele, aquela, aquilo}   \\
\rowcolor[HTML]{F5F5F5} 
{\color[HTML]{212529} These (plural)}   & {\color[HTML]{212529} Estes, estas}                     \\
\rowcolor[HTML]{F9F9F9} 
{\color[HTML]{212529} Those (plural)}  & {\color[HTML]{212529} Esses, essas, aqueles, aquelas}  \\ \bottomrule
\end{tabular}%
}
\end{table}

\vspace{0.5cm}

\begin{minipage}{.49\textwidth}
Quando algo está perto
\begin{itemize}
    \item This ou these
\end{itemize}
\end{minipage}
\begin{minipage}{.49\textwidth}
Quando algo está mais longe
\begin{itemize}
    \item That ou those
\end{itemize}
\end{minipage}

\end{frame}

%% --------------------------------------------------------

\begin{frame}{Pronomes indefinidos}

Os pronomes indefinidos são aqueles que substituem ou acompanham o substantivo de maneira imprecisa ou indeterminada.

\vspace{0.5cm}

Muitos dos pronomes indefinidos são formados com as palavras \textbf{some}, \textbf{any}, \textbf{no} e \textbf{every}.

\vspace{0.5cm}

Quando estão relacionados com pessoas, os pronomes apresentam a terminação: –\textit{body} ou –\textit{one}. 

Para coisas, a terminação é –\textit{thing}. 

Para lugares é –\textit{where}.

\end{frame}

%% --------------------------------------------------------

\begin{frame}{Pronomes indefinidos - some}

Sozinho, o termo some significa algum, alguns, um, uns, uma(s), alguma(s), algo, cerca de, certo(s), certa(s), um pouco de.

\vspace{0.25cm}

Quando acompanhados de sufixos a tradução pode ser diferente

\begin{table}[]
\centering
\label{tab:my-table5}
\resizebox{\textwidth}{!}{%
\begin{tabular}{@{}lll@{}}
\rowcolor[HTML]{3F6F96} 
{\color[HTML]{FFFFFF} \textbf{Palavra}} & {\color[HTML]{FFFFFF} \textbf{Tradução}} & {\color[HTML]{FFFFFF} \textbf{Exemplos}}                           \\
\rowcolor[HTML]{FFFFFF} 
{\color[HTML]{404040} Somebody}         & {\color[HTML]{404040} alguém}            & {\color[HTML]{404040} Somebody is missing.} \\
\rowcolor[HTML]{FFFFFF} 
{\color[HTML]{404040} Someone} &
  {\color[HTML]{404040} alguém} &
  {\color[HTML]{404040} Someone ate the last piece of pizza. } \\
\rowcolor[HTML]{FFFFFF} 
{\color[HTML]{404040} Something} &
  {\color[HTML]{404040} algo} &
  {\color[HTML]{404040} We are looking for something   to eat.} \\
\rowcolor[HTML]{FFFFFF} 
{\color[HTML]{404040} Somewhere} &
  {\color[HTML]{404040} em algum lugar} &
  {\color[HTML]{404040} Somewhere in Brazil.} \\
\rowcolor[HTML]{FFFFFF} 
{\color[HTML]{404040} Someway} &
  {\color[HTML]{404040} de alguma maneira} &
  {\color[HTML]{404040} I will get there someway. } \\ \bottomrule
\end{tabular}%
}
\end{table}

\vspace{0.25cm}

O \textbf{some} e seus derivados são utilizados em frases afirmativas. Nalguns casos, o some pode surgir em frases interrogativas.

\end{frame}

%% --------------------------------------------------------

\begin{frame}{Pronomes indefinidos - any}

O termo \textbf{any} significa: qualquer, quaisquer, algum, alguns, alguma(s), nenhum, nenhuma, um, uns, uma(s). No entanto, muitas palavras recebem sufixos

\begin{table}[]
\centering
\label{tab:my-table8}
\resizebox{\textwidth}{!}{%
\begin{tabular}{@{}lll@{}}
\rowcolor[HTML]{3F6F96} 
{\color[HTML]{FFFFFF} \textbf{Palavra}} &
  {\color[HTML]{FFFFFF} \textbf{Tradução}} &
  {\color[HTML]{FFFFFF} \textbf{Exemplos}} \\
\rowcolor[HTML]{FFFFFF} 
{\color[HTML]{404040} Anybody} &
  {\color[HTML]{404040} ninguém} &
  {\color[HTML]{404040} Can anybody help me with my homework?} \\
\rowcolor[HTML]{FFFFFF} 
{\color[HTML]{404040} Anyone} &
  {\color[HTML]{404040} qualquer um, ninguém} &
  {\color[HTML]{404040} I didn’t know anyone at the party.} \\
\rowcolor[HTML]{FFFFFF} 
{\color[HTML]{404040} Anything} &
  {\color[HTML]{404040} nada} &
  {\color[HTML]{404040} I didn't hear anything.} \\
\rowcolor[HTML]{FFFFFF} 
{\color[HTML]{404040} Anywhere} &
  {\color[HTML]{404040} qualquer lugar} &
  {\color[HTML]{404040} I would go anywhere with you.} \\
\rowcolor[HTML]{FFFFFF} 
{\color[HTML]{404040} Anyway} &
  {\color[HTML]{404040} de qualquer forma, jeito} &
  {\color[HTML]{404040} Anyway, we found a place to eat. } \\ \bottomrule
\end{tabular}%
}
\end{table}

\vspace{0.25cm}

O \textbf{any} e os outros pronomes indefinidos associados a ele são geralmente utilizados em negações ou perguntas.

\end{frame}

%% --------------------------------------------------------

\begin{frame}{Pronomes indefinidos - no e none}

O \textbf{no} é utilizado como pronome adjetivo. Já o \textbf{none} como pronome substantivo. 
\begin{itemize}
    \item A tradução de ambos é: nenhum, nenhuma
\end{itemize}

Quando acrescidos sufixos, a tradução muda

\begin{table}[]
\centering
\label{tab:my-table9}
\resizebox{\textwidth}{!}{%
\begin{tabular}{@{}lll@{}}
\rowcolor[HTML]{3F6F96} 
{\color[HTML]{FFFFFF} \textbf{Palavra}} & {\color[HTML]{FFFFFF} \textbf{Tradução}} & {\color[HTML]{FFFFFF} \textbf{Exemplos}}             \\
\rowcolor[HTML]{FFFFFF} 
{\color[HTML]{404040} Nobody} & {\color[HTML]{404040} ninguém} & {\color[HTML]{404040} Diana threw a party, but nobody showed up.}               \\
\rowcolor[HTML]{FFFFFF} 
{\color[HTML]{404040} No one} & {\color[HTML]{404040} ninguém} & {\color[HTML]{404040} I thought I heard someone, but   there was no one there.} \\
\rowcolor[HTML]{FFFFFF} 
{\color[HTML]{404040} None}             & {\color[HTML]{404040} nenhum, nenhuma}   & {\color[HTML]{404040} None of these apples is ripe.} \\
\rowcolor[HTML]{FFFFFF} 
{\color[HTML]{404040} Nothing}          & {\color[HTML]{404040} nada}              & {\color[HTML]{404040} I have nothing in my purse.}   \\
\rowcolor[HTML]{FFFFFF} 
{\color[HTML]{404040} Nowhere}          & {\color[HTML]{404040} em nenhum lugar}   & {\color[HTML]{404040} Where are you going? Nowhere.} \\ \bottomrule
\end{tabular}%
}
\end{table}

\vspace{0.25cm}

Ambos os pronomes são utilizados em sentenças que expressam negação.

\end{frame}

%% --------------------------------------------------------

\begin{frame}{Pronomes indefinidos - every}

O pronome \textbf{every} dependendo de seu contexto
\begin{itemize}
    \item Pode significar todo(s), toda(s) e cada
    \item Quando recebem o sufixo, a tradução é diferente
\end{itemize}

\begin{table}[]
\centering
\label{tab:my-table10}
\resizebox{\textwidth}{!}{%
\begin{tabular}{@{}lll@{}}
\rowcolor[HTML]{3F6F96} 
{\color[HTML]{FFFFFF} \textbf{Palavra}} & {\color[HTML]{FFFFFF} \textbf{Tradução}}   & {\color[HTML]{FFFFFF} \textbf{Exemplos}}                    \\
\rowcolor[HTML]{FFFFFF} 
{\color[HTML]{404040} Everybody} &
  {\color[HTML]{404040} toda a gente, todo o mundo} &
  {\color[HTML]{404040} Everybody I know prefers chocolate to vanilla.} \\
\rowcolor[HTML]{FFFFFF} 
{\color[HTML]{404040} Everyone}         & {\color[HTML]{404040} todos, todo o mundo} & {\color[HTML]{404040} Everyone wants to come to the party.} \\
\rowcolor[HTML]{FFFFFF} 
{\color[HTML]{404040} Everything}       & {\color[HTML]{404040} tudo}                & {\color[HTML]{404040} Everything is possible.}              \\
\rowcolor[HTML]{FFFFFF} 
{\color[HTML]{404040} Everywhere}       & {\color[HTML]{404040} em todos os lugares} & {\color[HTML]{404040} God is everywhere.}                   \\
\rowcolor[HTML]{FFFFFF} 
{\color[HTML]{404040} Every way} &
  {\color[HTML]{404040} de todo o jeito, todos os   sentidos} &
  {\color[HTML]{404040} The new system is functioning perfectly in every way.} \\ \bottomrule
\end{tabular}%
}
\end{table}

\end{frame}

%% --------------------------------------------------------

\begin{frame}{Outros pronomes indefinidos}

\begin{table}[]
\centering
\label{tab:my-table11}
\resizebox{\textwidth}{!}{%
\begin{tabular}{@{}lll@{}}
\rowcolor[HTML]{3F6F96} 
{\color[HTML]{FFFFFF} \textbf{Palavra}} &
  {\color[HTML]{FFFFFF} \textbf{Tradução}} &
  {\color[HTML]{FFFFFF} \textbf{Exemplos}} \\
\rowcolor[HTML]{FFFFFF} 
{\color[HTML]{404040} One} &
  {\color[HTML]{404040} algum, alguma, a gente, certo, um certo} &
  {\color[HTML]{404040} He is the one.} \\
\rowcolor[HTML]{FFFFFF} 
{\color[HTML]{404040} All} &
  {\color[HTML]{404040} tudo, todo(s), toda(s)} &
  {\color[HTML]{404040} All of these clothes need to be sold.} \\
\rowcolor[HTML]{FFFFFF} 
{\color[HTML]{404040} Much} &
  {\color[HTML]{404040} muito, muitas} &
  {\color[HTML]{404040} I spend much time reading.} \\
\rowcolor[HTML]{FFFFFF} 
{\color[HTML]{404040} Many} &
  {\color[HTML]{404040} muitos, muitas} &
  {\color[HTML]{404040} I have many t-shirts.} \\
\rowcolor[HTML]{FFFFFF} 
{\color[HTML]{404040} Little} &
  {\color[HTML]{404040} pouco, pouca} &
  {\color[HTML]{404040} I am a little upset.} \\
\rowcolor[HTML]{FFFFFF} 
{\color[HTML]{404040} Few} &
  {\color[HTML]{404040} poucos, poucas} &
  {\color[HTML]{404040} Few have ever seen this sculpture.} \\
\rowcolor[HTML]{FFFFFF} 
{\color[HTML]{404040} Fewer} &
  {\color[HTML]{404040} menos} &
  {\color[HTML]{404040} There will be fewer people at lunch than I expected.} \\
\rowcolor[HTML]{FFFFFF} 
{\color[HTML]{404040} Each} &
  {\color[HTML]{404040} cada} &
  {\color[HTML]{404040} Each person is different.} \\
\rowcolor[HTML]{FFFFFF} 
{\color[HTML]{404040} Such} &
  {\color[HTML]{404040} tal, tais} &
  {\color[HTML]{404040} I find such people very annoying.} \\
\rowcolor[HTML]{FFFFFF} 
{\color[HTML]{404040} Other} &
  {\color[HTML]{404040} outro(s), outra(s)} &
  {\color[HTML]{404040} I have other things to do.} \\
\rowcolor[HTML]{FFFFFF} 
{\color[HTML]{404040} Another} &
  {\color[HTML]{404040} outro, outra} &
  {\color[HTML]{404040} I would like another tea, please.} \\
\rowcolor[HTML]{FFFFFF} 
{\color[HTML]{404040} Either} &
  {\color[HTML]{404040} um ou outro, uma ou outra,   cada} &
  {\color[HTML]{404040} I like both pants. I would be happy with either.} \\
\rowcolor[HTML]{FFFFFF} 
{\color[HTML]{404040} Neither} &
  {\color[HTML]{404040} nem um(a) nem outro(a)} &
  {\color[HTML]{404040} Neither one of them understood what was happening.} \\
\rowcolor[HTML]{FFFFFF} 
{\color[HTML]{404040} Both} &
  {\color[HTML]{404040} ambos, ambas} &
  {\color[HTML]{404040} I have two sisters. I like both.} \\
\rowcolor[HTML]{FFFFFF} 
{\color[HTML]{404040} Enough} &
  {\color[HTML]{404040} bastante, suficiente} &
  {\color[HTML]{404040} That's enough!} \\
\rowcolor[HTML]{FFFFFF} 
{\color[HTML]{404040} Several} &
  {\color[HTML]{404040} vários, várias} &
  {\color[HTML]{404040} Several things have to be done this week.} \\ \bottomrule
\end{tabular}%
}
\end{table}

\end{frame}

\end{document}